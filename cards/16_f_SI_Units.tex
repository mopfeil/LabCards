	\begin{tikzpicture}
		\cardbackground{images/SI_Units.png}{0mm}
		\cardtypeSiUnit
		\cardtitle{Metre}
		\cardcontent{\fontsize{9}{10}\selectfont From 1889 the meter was defined by a prototype meter 	(\glqq Urmeter\grqq) and from 1960 by a special light wavelength. On 20 May 2019, World Metrology Day, a new fundamental revision by the General Conference on Weights and Measures came into force.
		}
		{A meter is defined as the length of the distance the light travels in a vacuum for a period of 		1/299,792,458 seconds.\\[5pt]				Dimension name: Length \\		Dimension symbol: L  \\		Size symbol: \textit{l}	\\		Unit symbol: m		\\[2pt]		$c = \frac{1}{299,792,458 } \; m/s$\\		$1m=\frac{1}{299,792,458 } \,s*c$	\\		}
		\cardprice{6}
		\cardborder
%		\carddebug
	\end{tikzpicture}
