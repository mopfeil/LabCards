	\begin{tikzpicture}
		\cardbackground{images/wien_bridge2.png}{0mm}
		\cardtypeMeas
		\cardtitle{Wien bridge}
		\cardcontent{\fontsize{9}{10}\selectfont A Wien bridge is a bridge circuit in which one bridge branch is formed by a bandpass and the other by a 2:1 voltage divider ($B=\frac{1}{3}U_{0}$). The Wien bridge is used for precision measurement of capacitance in terms of resistance and frequency. \\	
		}
		{The differential voltage U5 is evaluated, which shows a minimum at the frequency $f = 1/ 2\pi RC$ . There is also a phase shift from $-90^\circ$ to $+90^\circ$.\\ 		The picture shows two frequency responses: the sharp curve (black) for ideal components, the flat curve (blue) when R1 is increased by 5 \%: Even small tolerances worsen the quality factor.\\					Balance Condition:\\		R1 = R2; C1 = C2; R4= 2*R3\\[5pt]				$\omega^2 = \frac{1}{R1R2C1C2}$\\[5pt]		$\frac{C2}{C1} = \frac{R4}{R3}-\frac{R2}{R1}$		}
		\cardprice{13}
		\cardborder
%		\carddebug
	\end{tikzpicture}
