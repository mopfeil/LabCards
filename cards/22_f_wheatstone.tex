	\begin{tikzpicture}
		\cardbackground{images/wheatstone.png}{0mm}
		\cardtypeMeas
		\cardtitle{Wheatstone bridge}
		\cardcontent{The Wheatstone bridge is a circuit for measuring electrical resistances (DC resistance) and small ohmic resistance changes. The applications are mainly found in measurement and control technology.
		}
		{If the ratio of the voltage divider resistances is the same, then both points (A,B) have the same potential(balance condition: $\frac{R1}{R2}=\frac{R3}{R4}$). If a resistance is changed, a current flows from A to B or vice versa.\\[2pt]		e.g.: A fixed resistor is replaced by a semiconductor component. The semiconductor reacts to voltage changes, temperature, light or similar. In this way, changes due to a current flow or potential change between points A and B can be used for evaluation.\\[5pt]		$ U5 = U1-U3 = U0(\frac{R1}{R1+R2}-\frac{R3}{R3+R4})$		}
		\cardprice{12}
		\cardborder
%		\carddebug
	\end{tikzpicture}
