	\begin{tikzpicture}
		\cardbackground{images/step_response2.png}{0mm}
		\cardtypeOther
		\cardtitle{step response}
		\cardcontent{\fontsize{9}{10}\selectfont In electronic engineering and control theory, step response is the time behaviour of the outputs of a general system when its input is a step function. The unit step is commonly used to characterize a system’s response to sudden changes in its input.
		}
		{The picture shows the step response of an RC low pass. The step response always consists of a stationary part (1) and a transient part $(-e^{\frac{t}{\tau}})$. $\tau$ is called the time constant. It is the time it takes for the step response to rise to 63.2\% of its final value. The steady-state  error  is  the  error  after  the  transient  response  has  decayed  leaving only  the continuous response.  The error signal:  \\						$e(t)= U_{in}-U_{out} = 1-1+e^{-\frac{t}{\tau}}=e^{-\frac{t}{\tau}}$\\				The step response of the system is:\\				$U(t)=U_{max}(1-e^{-\frac{t}{\tau}})$					}
		\cardprice{49}
		\cardborder
%		\carddebug
	\end{tikzpicture}
