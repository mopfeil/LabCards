%   COMMANDS ZUM ZUSAMMENBAUEN DER KARTEN
%   ---------------------------------------


%   TikZ/PGF Settings für die Karten
\pgfmathsetmacro{\cardwidth}{8.4}
\pgfmathsetmacro{\cardheight}{12.6}
\pgfmathsetmacro{\imagewidth}{0.9*\cardwidth}
\pgfmathsetmacro{\imageheight}{0.75*\cardheight}
\pgfmathsetmacro{\stripwidth}{0.7}
\pgfmathsetmacro{\strippadding}{0.2}
\pgfmathsetmacro{\textpadding}{0.1}
\pgfmathsetmacro{\titley}{\cardheight-\strippadding-1.5*\textpadding-0.5*\stripwidth}


%   Formen der einzelnen Kartenelemente/-bestandteile
\def\shapeCard{(0,0) rectangle(\cardwidth,\cardheight)}

\def\shapeLeftStripLong{(\strippadding,-0.2) rectangle(\strippadding+\stripwidth,0.65*\cardheight-\strippadding-\strippadding-1)}

\def\shapeLeftStripShort{(\strippadding,\cardheight-\strippadding-1) rectangle (\strippadding+\stripwidth,\cardheight+0.2)}

\def\shapeRightStripShort{(\cardwidth-\stripwidth-\strippadding,\cardheight-\strippadding-1) rectangle (\cardwidth-\strippadding,\cardheight+0.2)}

\def\shapeTitleArea{(2*\strippadding+\stripwidth,\cardheight-\strippadding) rectangle (\cardwidth-2*\strippadding-\stripwidth,\cardheight-2*\stripwidth)}

\def\shapeContentArea{(2*\strippadding+\stripwidth,0.65*\cardheight-\strippadding-\strippadding-1) rectangle (\cardwidth+0.2,-0.2)}


%   Stylings für die Elemente definieren
\tikzset{
    %   runde Ecken für die Karten
    cardcorners/.style={
        rounded corners=0.5cm
    },
    %   runde Ecken für die "Fähnchen"
    elementcorners/.style={
        rounded corners=0.1cm
    },
    %   Schlagschatten für die "Fähnchen"
    stripshadow/.style={
        drop shadow={
            opacity=.5,
            color=black
        }
    },
    %   Style für die "Fähnchen"
    strip/.style={
        elementcorners,
        stripshadow
    },
    %   Bild für das Kartenmotiv
    cardimage/.style n args={2}{
    	path picture={
    		\node[below=15mm+ #2] at (0.5*\cardwidth,\cardheight) {
    			\includegraphics[width=\imagewidth cm]{#1}
    		};
    	}
    },
}

%   TikZ-Raster
\newcommand{\carddebug}{
\draw [step=1,help lines] (0,0) grid (\cardwidth,\cardheight);
}

%   Rahmen der Karte
\newcommand{\cardborder}{
\draw[lightgray,cardcorners] \shapeCard; % Kartenrand in hellgrau
}

%   Hintergrund der Karte (Außenlinien der Karte mit runden Ecken und Bild)
\newcommand{\cardbackground}[2]{
\draw[cardcorners, cardimage={#1}{#2}] \shapeCard;
}

%   Kategorie der Karte
\newcommand{\cardtype}[3]{
    %   First we fill the intersecting area
    %   The \clip command does not allow options, therefore 
    %   we have to use a scope to set the even odd rule.
    \begin{scope}[even odd rule]
        %   Define a clipping path. All paths outside shapeCard will
        %   be cut because the even odd rule is set.
        \clip[cardcorners] \shapeCard;
        % Fill shapeLeftStripLong and shapeLeftStripShort.
        %   Since the even odd rule is set, only the card will be filled.
        \fill[strip,#1] \shapeLeftStripLong node[rotate=90,above left=0.9mm, font=\normalsize] {
            \color{white} \uppercase{#2}
        };
        \fill[strip,#1] \shapeLeftStripShort;
    \end{scope}

    \node at (\strippadding+\stripwidth-0.28,\cardheight-\strippadding-\strippadding-0.37) {\color{white}#3};
}

%\newcommand{\cardtypeCircuit}{\cardtype{circuitbg}{Circuit Knowome}{\hspace{-1mm}\Large\floweroneleft}}
%\newcommand{\cardtypeCircuit}{\cardtype{circuitbg}{Circuit Knowome}{\includegraphics[width= 25pt,  decodearray={0.5 0.1}]{images/schaltzeichen-diode.png}}}
%\newcommand{\cardtypeCircuit}{\cardtype{circuitbg}{Circuit Knowome}{\hspace{-1mm}\huge\AC}}
%\newcommand{\cardtypeFormal}{\cardtype{formalbg}{Formal Knowome}{\hspace{-1mm}\Large\lefthand}}
%\newcommand{\cardtypeCode}{\cardtype{codebg}{Code Knowome}{\hspace{-1mm}\LARGE\aldineleft}}
%\newcommand{\cardtypeOther}{\cardtype{otherbg}{Other Knowome}{\hspace{-1.4mm}\huge\ding{78}}}

\newcommand{\cardtypeProtect}{\cardtype{violet}{Protection Circuit}{\hspace{-1mm}\huge$\nabla$}}
\newcommand{\cardtypeFilter}{\cardtype{orange}{Filter Circuit}{\hspace{-1mm}\huge\VHF}}
\newcommand{\cardtypeTTL}{\cardtype{red}{DTL/TTL Circuit}{\hspace{-1mm}\huge\AC}}
\newcommand{\cardtypeFET}{\cardtype{lightblue}{FET Circuit}{\hspace{-1mm}\huge\textbrokenbar}}
\newcommand{\cardtypeOPV}{\cardtype{blue}{OPV Circuit}{\hspace{-1mm}\huge$\rhd$}}
\newcommand{\cardtypeCircuit}{\cardtype{green}{Circuit Knowome}{\hspace{-1mm}\huge\lightning}}
\newcommand{\cardtypeBack}{\cardtype{violet}{ \tiny Ravensburg-Weingarten University}{\hspace{-1mm} \tiny RWU}}
\newcommand{\cardtypeStatistics}{\cardtype{red}{Statistics}{\hspace{-1mm}\huge$\spadesuit$}}
\newcommand{\cardtypeMeas}{\cardtype{yellow}{Measurement}{\hspace{-1mm}\huge$\nabla$}}
\newcommand{\cardtypeOther}{\cardtype{black}{Other}{\hspace{-1mm}\huge$\nabla$}}
\newcommand{\cardtypeSiUnit}{\cardtype{violet}{Other}{\hspace{-1mm}\huge$\nabla$}}
%   Titel der Karte
\newcommand{\cardtitle}[1]{
    %\draw[pattern=soft crosshatch,rounded corners=0.1cm] \shapeTitleArea;
    \fill[elementcorners,titlebg,opacity=.85] \shapeTitleArea;
    \node[text width=5.75cm] at (0.5*\cardwidth,\titley) {
        \begin{center}
           \color{white}\uppercase{\normalsize #1}
        \end{center}
    };
}

%   Inhalt der Karte
\newcommand{\cardcontent}[2]{
    \begin{scope}[even odd rule]
        \clip[cardcorners] \shapeCard;
        \fill[elementcorners,contentbg] \shapeContentArea;
    \end{scope}
    \node[below right, text width=(\cardwidth-2*\strippadding-\stripwidth-2*\textpadding-0.3)*1cm] at (2*\strippadding+\stripwidth+\textpadding,0.5*\cardheight-\textpadding+0.5) {
        {\baselineskip=8pt \normalsize #1 \par }
    };

    \node[below right, text width=(\cardwidth-2*\strippadding-\stripwidth-2*\textpadding-0.3)*1cm] at (2*\strippadding+\stripwidth+\textpadding,4.2) {
        \vrule width \textwidth height 2pt \\[-2pt] 	% schwarze Trennlinie
        \vspace{0.2cm}
        {\baselineskip=6pt \scriptsize #2 \par}
    };
}

%   Preis der Karte
\newcommand{\cardprice}[1]{
    \begin{scope}[even odd rule]
        \clip[cardcorners] \shapeCard;
        \fill[strip,pricebg] \shapeRightStripShort;
    \end{scope}
    \node at (\cardwidth-0.5*\stripwidth-\strippadding,\titley-0.1) {\color{black}#1};
}